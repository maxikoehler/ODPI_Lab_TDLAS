%!TEX root = ../main.tex

\chapter{Introduction}
\label{chap:intro}

A rather simple, but yet effective optical technique for analyzing and determining the species, concentration, pressure, temperatures or density of materials is the Tunable Laser Absorption Spectroscopy (TDLAS). This optical method relies on the specific and characteristic light absorption of substances. It used a manipulated laser to generate a bandwidth of wavelength. With that one can have a look at complete spectra and the absorption behavior over them. It can be grouped into the laser spectroscopy methods. Specific applications are particle technology (determining particle densities or concentrations), environmental technology, material composition or other technical fields.

This paper is documenting and discussing an experiment, more specifically a laboratory course, with this measurement method. It is carried out at the Institute for Engineering Thermodynamics (LTT) of the FAU Erlangen-Nuremberg at its site in Erlangen-Tennenlohe.

The assignment, description of the equipment and procedure and further details about the Lab Course are described in the given handbook \autocite{klevanskyTDLASTunableLaser2021}.

% \vspace{1cm}

% \commenting{
%     \begin{itemize}
        % \item Cover page\\
        % The title of the experiment, the names, courses of study, matriculation numbers of all
        % participants, the group number and the name of the supervisor must be noted on the
        % cover sheet..
        % \item Outline\\
        % Before the actual evaluation of the experiment, a table of contents (with page numbers)
        % is to be inserted.
        % \item Introduction\\
        % The introduction should briefly show the subject matter and the aim of the practical
        % experiment as well as the relevance of measurement technology for industrial
        % applications.
        % \item Theoretical background\\
        % In the theoretical background, the underlying physical relationships of absorption
        % spectroscopy should be briefly explained..
        % \item Measurement procedure\\
        % In the chapter "Experimental procedure", the experimental set-up of the practical course
        % should be briefly presented and explained, and the evaluation strategy should be
        % presented in detail and comprehensibly (state formulae; explain calculation steps).
%         \item Results\\
%         In the results section, the results are to be presented appropriately and discussed briefly
%         in each case. Make sure that the dark signals of channel "a" and "b" and the measured
%         calibration data are taken into account.
%         \begin{enumerate}
%             \item Calibration curve for the two measuring channels.\\
%             -> Presentation of the calibration curve in an xy-diagramm (x = wavelength; y = calibration factor) additional value table for the wavelength-dependent calibration factors. Discussion of further influences and suggestions for improvement.
%             \item Pure substance spectra and reference spectrum (see 4.1)\\
%             -> Presentation of the 2 absorption spectra of the pure substances in an xy-diagramm (x = wavelength; y = calibration factor) and labelling of the wavelengths for determination of the extinction coefficient..
%             \item Determination of the extinction coefficient of [EMIM][EtSO4] and DMSO for positions 2, 5, 7 und 9 (siehe 4.2)\\
%             -> Tabulation of the results
%             \item Discussion of the measurement uncertainty for [EMIM][EtSO4] and suggestions for improvement.
%             \item Discussion of errors.
%         \end{enumerate}
%         \item Conclusion
%     \end{itemize}
% }
