%!TEX root = ../main.tex

\usetikzlibrary{positioning,calc,shapes,arrows,shapes.multipart}

% circuitikz: creating a bus
\tikzset{bus/.style={fullgeneric, %
        bipoles/fullgeneric/width=0.02, bipoles/fullgeneric/height=#1
    },
    bus/.default=3
}
\newcommand{\bushere}[3]{% length, text above, text below
    % optional arguments do not work in paths
    %
    % starting point; draw an edge and then two nodes
    % save the position
    coordinate(tmp)
    % go up and do an edge down
    ++(0,#1) node[anchor=base]{#2} edge[ultra thick] ++(0,{-2*#1})
    % edges do not move the current point, go down to position the node
    ++(0,{-2*#1}) node[below]{#3}
    % go back to where we started
    (tmp)
}

% program plan
\tikzset{
   papDecision/.style = {
         diamond,
         draw, 
         text width = 20 mm, 
         align = center, 
         text badly centered,
         inner sep = 1 pt,
         font=\ttfamily\footnotesize,
         %line width = 1,
         minimum width = 30mm,
         minimum height = 7mm,
      },
   papStart/.style = {
         rectangle,
         draw, 
         align = center, 
         text width = 3cm, 
         text badly centered,
         inner sep = 4 pt,
         rounded corners=10pt,
         font=\ttfamily\footnotesize,
         %line width = 1,
         minimum width = 30mm,
         minimum height = 7mm,
      },
   papEnd/.style = {
         rectangle,
         draw, 
         align = center, 
         text width = 3cm, 
         text badly centered,
         inner sep = 4 pt,
         rounded corners=10pt,
         font=\ttfamily\footnotesize,
         %line width = 1,
         minimum width = 30mm,
         minimum height = 7mm,
      },
   papData/.style = {
         trapezium,
         draw, 
         align = center, 
         text width = 20 mm, 
         text badly centered,
         inner sep = 4 pt,
         trapezium left angle=70,
         trapezium right angle=110,
         font=\ttfamily\footnotesize,
         %line width = 1,
         minimum width = 30mm,
         minimum height = 7mm,
      },
   papPredProc/.style = {
         draw,
         rectangle split,
         rectangle split horizontal,
         rectangle split parts = 3,
         rectangle split empty part width=-8pt,
         align = center, 
 %       text width = 4.5 em, 
         text badly centered,
 %        inner sep = 4 pt,
         font=\ttfamily\footnotesize,
         %line width = 1,
         minimum width = 30mm,
         minimum height = 7mm,
      },
   papProcess/.style = {
         rectangle,
         draw,
         align = center, 
         text width = 3cm, 
         text badly centered,
         %inner sep = 2 pt,
         font=\ttfamily\footnotesize,
         %line width = 1,
         minimum width = 30mm,
         minimum height = 7mm,
      },
   papLine/.style = {
         draw,
         -stealth,
         font=\ttfamily\footnotesize,
         %line width = 1,
      },
}
\newcommand{\papYes}{ja}
\newcommand{\papNo}{nein}